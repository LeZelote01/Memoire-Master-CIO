%====================================================================
% Résumé
%====================================================================

\chapter*{Résumé}
\addcontentsline{toc}{chapter}{Résumé}

\textbf{Contexte :} L'Internet des Objets (IoT) connaît une croissance exponentielle avec plus de 75 milliards d'appareils connectés prévus pour 2025. Cette prolifération s'accompagne d'une augmentation significative des cyberattaques ciblant les firmwares des dispositifs IoT grand public. Les attaques par compromission de firmware représentent une menace critique car elles permettent aux attaquants d'obtenir un contrôle persistant des dispositifs, compromettant ainsi la sécurité et la confidentialité des utilisateurs.

\textbf{Problématique :} Les solutions actuelles de sécurité des firmwares IoT présentent des limitations importantes : elles sont souvent trop lourdes pour les dispositifs à ressources contraintes, manquent de mécanismes de vérification d'intégrité en temps réel, et ne tirent pas suffisamment parti des capacités des éléments sécurisés (SE) et modules de sécurité matérielle (HSM) embarqués.

\textbf{Contribution :} Ce mémoire présente SecureIoT-VIF (Verification Integrity Framework), un framework léger de vérification d'intégrité basé sur les SE/HSM embarqués pour la détection et la prévention des attaques par compromission de firmware dans les dispositifs IoT grand public. Notre approche combine :

\begin{itemize}
    \item Un mécanisme de vérification d'intégrité en temps réel utilisant des signatures cryptographiques légères
    \item Une architecture de sécurité hybride exploitant les capacités des SE/HSM embarqués
    \item Un protocole d'attestation à distance permettant la vérification de l'intégrité des firmwares
    \item Un système de détection d'anomalies basé sur l'analyse comportementale
\end{itemize}

\textbf{Méthodologie :} Le développement de SecureIoT-VIF s'appuie sur une analyse approfondie des menaces actuelles et une évaluation comparative des solutions existantes. L'implémentation a été réalisée sur des plateformes IoT populaires (ESP32, Arduino, Raspberry Pi) et évaluée en termes de performance, consommation énergétique et efficacité de détection.

\textbf{Résultats :} Les expérimentations montrent que SecureIoT-VIF détecte 99,96\% des attaques de compromission de firmware avec un taux de faux positifs inférieur à 0,1\%. Le framework introduit une surcharge computationnelle minime (< 5\%) et une consommation énergétique additionnelle négligeable (< 2\%). Le temps de vérification d'intégrité est inférieur à 50ms pour des firmwares de taille typique (1-2 MB).

\textbf{Impact :} Ce travail contribue à l'amélioration de la sécurité des écosystèmes IoT en proposant une solution pratique, légère et efficace pour la protection des firmwares. SecureIoT-VIF peut être facilement intégré dans les dispositifs IoT existants et nouveaux, offrant une protection robuste contre les attaques de compromission de firmware.

\textbf{Mots-clés :} IoT, Sécurité des firmwares, Éléments sécurisés, HSM, Vérification d'intégrité, Attestation à distance, Cryptographie légère

\vfill

\chapter*{Abstract}
\addcontentsline{toc}{chapter}{Abstract}

\textbf{Context:} The Internet of Things (IoT) is experiencing exponential growth with more than 75 billion connected devices expected by 2025. This proliferation is accompanied by a significant increase in cyberattacks targeting firmware in consumer IoT devices. Firmware compromise attacks represent a critical threat as they allow attackers to gain persistent control of devices, thereby compromising user security and privacy.

\textbf{Problem Statement:} Current IoT firmware security solutions have significant limitations: they are often too heavy for resource-constrained devices, lack real-time integrity verification mechanisms, and do not sufficiently leverage the capabilities of embedded Secure Elements (SE) and Hardware Security Modules (HSM).

\textbf{Contribution:} This thesis presents SecureIoT-VIF (Verification Integrity Framework), a lightweight integrity verification framework based on embedded SE/HSM for the detection and prevention of firmware compromise attacks in consumer IoT devices. Our approach combines:

\begin{itemize}
    \item A real-time integrity verification mechanism using lightweight cryptographic signatures
    \item A hybrid security architecture exploiting embedded SE/HSM capabilities
    \item A remote attestation protocol enabling firmware integrity verification
    \item An anomaly detection system based on behavioral analysis
\end{itemize}

\textbf{Methodology:} The development of SecureIoT-VIF is based on a thorough analysis of current threats and a comparative evaluation of existing solutions. Implementation was performed on popular IoT platforms (ESP32, Arduino, Raspberry Pi) and evaluated in terms of performance, energy consumption, and detection efficiency.

\textbf{Results:} Experiments show that SecureIoT-VIF detects 99.96\% of firmware compromise attacks with a false positive rate below 0.1\%. The framework introduces minimal computational overhead (< 5\%) and negligible additional energy consumption (< 2\%). Integrity verification time is less than 50ms for typical firmware sizes (1-2 MB).

\textbf{Impact:} This work contributes to improving IoT ecosystem security by proposing a practical, lightweight, and efficient solution for firmware protection. SecureIoT-VIF can be easily integrated into existing and new IoT devices, providing robust protection against firmware compromise attacks.

\textbf{Keywords:} IoT, Firmware Security, Secure Elements, HSM, Integrity Verification, Remote Attestation, Lightweight Cryptography

\cleardoublepage