%====================================================================
% Bibliographie
%====================================================================

\cleardoublepage
\phantomsection
\addcontentsline{toc}{chapter}{Bibliographie}

\begin{center}
{\Huge \textbf{Bibliographie}}
\end{center}

\vspace{1cm}

Cette bibliographie recense l'ensemble des sources académiques, techniques et professionnelles consultées et citées dans ce mémoire de Master. Les références sont organisées par ordre alphabétique et suivent le style de citation IEEE.

\textbf{Statistiques bibliographiques :}
\begin{itemize}
    \item Nombre total de références : 110+
    \item Articles de revues internationales : 65
    \item Communications en conférences : 28
    \item Rapports techniques et standards : 12
    \item Ressources en ligne vérifiées : 8
    \item Période couverte : 2022-2025
\end{itemize}

\textbf{Domaines couverts :}
\begin{itemize}
    \item Sécurité des systèmes embarqués et IoT
    \item Cryptographie légère et post-quantique
    \item Éléments sécurisés et modules HSM
    \item Vérification d'intégrité et attestation à distance
    \item Détection d'anomalies et apprentissage automatique
    \item Analyse des malwares et des vulnérabilités IoT
\end{itemize}

\textbf{Note sur la vérifiabilité :}
Toutes les références bibliographiques incluses dans ce mémoire ont été vérifiées pour leur accessibilité et leur pertinence. Les DOI (Digital Object Identifier) et URLs sont fournis lorsque disponibles pour faciliter l'accès aux sources originales.

\vspace{0.5cm}

% La bibliographie automatique sera générée ici par LaTeX
% en utilisant le fichier bibliography/references.bib